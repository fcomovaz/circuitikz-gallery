\begin{circuitikz}[circuitikz/straight=true,
    american,
    full diodes,
    circuitikz/bipoles/cuteswitch/thickness=0.3,
    circuitikz/diodes/scale=0.5,
    circuitikz/resistors/scale=0.7,
    circuitikz/capacitors/scale=0.7,
    circuitikz/diodes/scale=0.5,
    circuitikz/capacitors/thickness=4,]
    \node [transformer](T){};
    \path (T.A1) node[ocirc]{} (T.A2) node[ocirc]{};
    \draw (T.B1) to [short] ++(2,0) coordinate(top bridge);
    \draw (T.A1) to [open, v=$V_{in}$] (T.A2);
    % bridge
    % find the center of the bridge; cross the top bridge with
    % the center of the trafo
    \coordinate (center bridge) at (T.center -| top bridge);
    % now the coordinates to the right, bottom and left bridge
    % this uses the calc library of tikz
    \coordinate (right bridge) at ($(center bridge)!1!-90:(top bridge)$);
    \coordinate (left bridge) at ($(center bridge)!1!90:(top bridge)$);
    \coordinate (bottom bridge) at ($(center bridge)!1!-180:(top bridge)$);
    % draw the bridge
    \draw (top bridge) to [D, *-*] (right bridge)
    to [D, *-*, invert] (bottom bridge)
    to [D, *-*, invert] (left bridge)
    to [D, *-*] (top bridge);
    % the rest of the circuit
    \node [jump crossing](X) at (left bridge |- T.B2) {};
    \draw (T.B2) -- (X.west) (X.east) -- (bottom bridge);
    % notice that if you change the "-2" here then the rest of the circuit
    % will adapt, thank to using cross-coordinates and relative movements!
    % The same will happens to the numbers in the ++(...) coordinates.
        % ok, the rest of the circuit now
    \draw (right bridge) to[R=$R_s$,a=$10\Omega$, -] ++(4,0) coordinate(rs);
    \draw (rs) to[curved capacitor=$100\mu F$,a=$C$] ++(0,-2) coordinate(c2l);
    \draw (left bridge) -- (X.north) (X.south) |- (c2l);
    \draw (rs) to[short] ++(2.5,0) coordinate(r);
    \draw (r) to[R,a=$R$,l=$10k\Omega$] ++(0,-2) coordinate(_r);
    \draw (_r) to [short] ++(-3,0);
    \draw (r) to[short] ++(2,0) coordinate(vou1);
    \draw (_r) to[short] ++(2,0) coordinate(vou2);
    \draw (vou1) to[open,o-o] (vou2);
\end{circuitikz}